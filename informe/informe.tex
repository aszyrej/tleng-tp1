\documentclass[11pt, a4paper,spanish]{article}
\usepackage[spanish,activeacute]{babel}
\usepackage[utf8]{inputenc}
\usepackage{caratula}
%\usepackage[ruled,vlined,boxed,commentsnumbered,linesnumbered]{algorithm2e}
\usepackage{graphicx}
\usepackage{fancyhdr}
\usepackage{amsfonts}
\usepackage{amssymb}
\usepackage{amsmath}
\usepackage{hyperref}
\usepackage{float}
\usepackage{lastpage}
\usepackage{ulem}
\usepackage{listings}

\usepackage[ruled,vlined,boxed,commentsnumbered]{algorithm2e}

% Encabezado y Pie de pagina
\pagestyle{fancy}
\fancyhf{}
%\lfoot{\rightmark}
\lfoot{Teor\'ia de Lenguajes}
\rhead{Trabajo Pr\'actico - Compositor de f\'ormulas matem\'aticas}
\rfoot{\thepage\ de \pageref{LastPage}}


\begin{document}
% ------------ CARATULA ------------
    
   \materia{Trabajo Práctico: \\ Compositor de f\'ormulas matem\'aticas}
    \submateria{Teor\'ia de Lenguajes}

 

    \titulo{\large Resumen:}
    \subtitulo{\large
    Este trabajo consiste en el armado de un archivo SVG (f\'ormula) en base a un input, una cadena perteneciente al lenguaje de una gram\'atica ambigua definida en el enunciado.}
    

    \integrante{Santos, Marti\'in}{413/11}{martin.n.santos@gmail.com}

    \integrante{Szyrej, Alexander}{642/11}{alexander.szyrej@gmail.com}

    \integrante{Nu\~nez Morales, Carlos Daniel}{732/08}{cdani.nm@gmail.com}

% Keywords
    \director{Gram\'atica, Parsers, LALR, AST, SVG, LATEX}{}

	\maketitletxtlogo

	\pagebreak

	



% ------------ INDICE ------------

    \thispagestyle{empty}
    \tableofcontents
    \pagebreak

% ------------ INTRODUCCION ------------

	\begin{section}{Introduccion}
Este trabajo consta de dos etapas: Tokenizaci\'on y parseo del input por un lado (donde tambi\'en se incluye la desambiguaci\'on de la gram\'atica presentada), y por otro lado la generaci\'on del archivo de output esperado.
Para la primera parte se utiliz\'o el conjunto de herramientas de generaci\'on de c\'odigo python ply. El mismo cuenta con dos m\'odulos, ambos utilizados: Ley, para el proceso de tokenizaci\'on y Yacc, para parsear la gram\'atica.
Para la segunda parte, se cuenta con una estructura \'arbol, instanciada en la etapa de parsing del input, construida para simplificar el armado del archivo SVG.

\end{section}


% ------------ DESARROLLO ------------

	\begin{section}{Parte I: Tokenizacion y Parseo}

\begin{subsection}{Desambiguando la gram\'atica}
Utilizamos los m\'odulos de Lex y Yacc para esta primera parte de tokenizaci\'on y parsing. Puesto que el parser usado por el m\'odulo Yacc utiliza la t\'ecnica LALR, nos enfocamos en un primer momento en desambiguar la gram\'atica para que al momento de construir la tabla de $Action$ LALR no haya conflictos.

Gram\'atica inicial G:
\newline E $\rightarrow$ E E 
\newline $|$ E \^{} E
\newline $|$ E \_ E
\newline $|$ E \^{} E \_ E
\newline $|$ E \_ E \^{} E
\newline $|$ E / E
\newline $|$ ( E )
\newline $|$ \{ E \}
\newline $|$ c
  
~

Primeramente consideramos el orden de presedencia dispuesto en el enunciado. Luego la gram\'atica G1 se rige sobre la siguiente tabla: 
  
~

\begin{tabular}{ c | c | c }
  			
  Caracter & Presedencia & Asociatividad \\
  \hline
  / & 1 & Izq \\
  . & 2 & Izq \\
  \^{} & 3 & - \\
  \_ & 3 & - \\
  () & 4 & - \\
  \{\} & 4 & - \\
  
\end{tabular}
  
~

Luego G1 qued\'o determinada por el siguiente conjunto de producciones:
\newline \underline{E $\rightarrow$ E / T $|$ T}
\newline \underline{T $\rightarrow$ T F $|$ F}
\textbf{\newline F $\rightarrow$ I \^{} G $|$ G $|$ I \_ H $|$ H
\newline H $\rightarrow$ I \^{} I $|$ I
\newline G $\rightarrow$ I \_ I $|$ I}
\newline I $\rightarrow$ (E) $|$ \{E\} $|$ c
  
~

Notar que las producciones subrayadas son recursivas a izquierda, lo cual genera conflictos. 
Las producciones en negrita generaron otros problemas. En estas se debe factorizar a izquiera para no tener conflictos.

Finalmente la gram\'atica qued\'o de la pinta G2:
\newline E $\rightarrow$ T A
\newline A $\rightarrow$ / T A $|$ lambda
\newline T $\rightarrow$ F B
\newline B $\rightarrow$ F B $|$ lambda
\newline F $\rightarrow$ I G
\newline G $\rightarrow$ \^{} I H $|$ \_ I L $|$ lambda
\newline H $\rightarrow$ \_ I $|$ lambda
\newline L $\rightarrow$ \^{} I $|$ lambda
\newline I $\rightarrow$ (E) $|$ \{E\} $|$ c
  
~

donde la recursion a izquierda fue eliminada introduciendo las producciones de A y B, y los conflictos de F G y H fueron resueltos subiendo los casos de G y H hasta F y tomando factor com\'un a izquierda.

\end{subsection}
\end{section}

\begin{section}{Parte II: Construcci\'on SVG}

\end{section}


	\pagebreak

% ------------ RESULTADOS ------------

	%\begin{section}{Modelo Entidad Relaci\'on}

\begin{subsection}{Modelo Relacional}

\noindent \textbf{Partido Pol\'itico} $\lbrace$ \underline{idPartido}, Nombre $\rbrace$ \\
PK = CK = $ \lbrace $ idPartido $ \rbrace $ \\
\\
\textbf{Votante} $ \lbrace $ \underline{DNI}, Nombre, IdTerritorio, tipo $ \rbrace $\\
PK = CK = $ \lbrace $ DNI $ \rbrace $\\
FK = $ \lbrace $ idTerritorio $ \rbrace $\\
\\
\textbf{Camioneta} $ \lbrace $ \underline{Patente}, DNI $ \rbrace $\\
PK = CK = $ \lbrace $ Patente  $ \rbrace $\\
FK = $ \lbrace $ DNI $ \rbrace $\\
\\
\textbf{Centro} $ \lbrace $ \underline{idCentro}, direcci\'on, patente, idTerritorio $ \rbrace $ \\
PK = CK = $ \lbrace $ idCentro $ \rbrace $ \\
FK = $ \lbrace $ Patente, idTerritorio $ \rbrace $\\
\\
\textbf{Mesa} $ \lbrace $ \underline{IdMesa}, idCentro $ \rbrace $ \\
PK = CK = $ \lbrace $ IdMesa $ \rbrace $ \\
FK = $ \lbrace $ idCentro $ \rbrace $\\
\\
\textbf{Contiene} $ \lbrace $ \underline{IdElecci\'on, idMesa}, DNIPresidente, DNIVice, DNITecnico $ \rbrace $ \\
PK = CK = $ \lbrace $ $ \langle $ IdElecci\'on, idMesa $ \rangle $ $ \rbrace $ \\
FK = $ \lbrace $ IdElecci\'on, idMesa, DNIPresidente, DNIVice, DNITecnico $ \rbrace $\\
\\
\textbf{Votaci\'onEleccion} $ \lbrace $ \underline{idEleccion}, fecha, tipo $ \rbrace $ \\
PK = CK = $ \lbrace $ idEleccion $ \rbrace $ \\
\\
\textbf{esFiscal} $ \lbrace $ \underline{DNI, IdElecci\'on, idMesa}, idPartido $ \rbrace $ \\
PK = CK = $ \lbrace $ $ \langle $ DNI, IdElecci\'on, idMesa $ \rangle $ $ \rbrace $ \\
FK = $ \lbrace $ DNI, IdElecci\'on, idMesa, idPartido $ \rbrace $\\
\\
\textbf{Vota} $ \lbrace $ \underline{DNI, IdElecci\'on, idMesa}, Hora $ \rbrace $ \\
PK = CK = $ \lbrace $ $ \langle $ DNI, IdElecci\'on, idMesa $ \rangle $ $ \rbrace $ \\
FK = $ \lbrace $ DNI, IdElecci\'on, idMesa $ \rbrace $\\
\\
\textbf{VotacionCandidato} $ \lbrace $ \underline{IdElecci\'on}, idCargo, idTerritorio $ \rbrace $ \\
PK = CK = $ \lbrace $ IdElecci\'on $ \rbrace $ \\
FK = $ \lbrace $ IdElecci\'on, idCargo, idTerritorio $ \rbrace $\\
\\
\textbf{VotacionConsultaPopular} $ \lbrace $ \underline{IdElecci\'on}, idConsulta $ \rbrace $ \\
PK = CK = $ \lbrace $ IdElecci\'on $ \rbrace $ \\
FK = $ \lbrace $ idConsulta $ \rbrace $\\
\\
\textbf{ConsultaPopular} $ \lbrace $ \underline{IdConsulta}, Descripci\'on, CantVotosAFvor$ \rbrace $ \\
PK = CK = $ \lbrace $ IdElecci\'on $ \rbrace $ \\
\\
\textbf{Voto} $ \lbrace $ \underline{IdVoto,IdElecci\'on, idMesa}, tipo $ \rbrace $ \\
PK = CK = $ \lbrace $ $ \langle $ IdVoto, IdElecci\'on, idMesa $ \rangle $  $ \rbrace $ \\
FK = $ \lbrace $ $ \langle $ IdElecci\'on, idMesa $ \rangle $ $ \rbrace $\\
\\
\textbf{VotoConsultaPopular} $ \lbrace $ \underline{IdVoto}$ \rbrace $ \\
PK = CK = $ \lbrace $  IdVoto $ \rbrace $ \\
FK = $ \lbrace $  $ \rbrace $\\
\\
\textbf{VotoCandidato} $ \lbrace $ \underline{IdVoto}, DNI$ \rbrace $ \\
PK = CK = $ \lbrace $  IdVoto $ \rbrace $ \\
FK = $ \lbrace $ DNI $ \rbrace $\\
\\
\textbf{SePostulaA} $ \lbrace $ \underline{DNI, IdElecci\'on}, idPartido, cantVotos $ \rbrace $ \\
PK = CK = $ \lbrace $ $ \langle $ DNI, IdElecci\'on $ \rangle $ $ \rbrace $ \\
FK = $ \lbrace $ DNI, IdElecci\'on, idPartido $ \rbrace $\\
\\
\textbf{Cargo} $ \lbrace $ \underline{idCargo}, Nombre, Período $ \rbrace $ \\
PK = CK = $ \lbrace $ idCargo $ \rbrace $ \\
FK = $ \lbrace $ Nombre $ \rbrace $\\
\\
\textbf{RigePara} $ \lbrace $ \underline{idCargo, idTerritorio} $ \rbrace $ \\
PK = CK = $ \lbrace $ $ \langle $ idCargo, idTerritorio $ \rangle $ $ \rbrace $ \\
FK = $ \lbrace $ idCargo, idTerritorio $ \rbrace $\\
\\
\textbf{Territorio} $ \lbrace $ \underline{idTerritorio}, Nombre, idSupreTerritorio $ \rbrace $ \\
PK = CK = $ \lbrace $ idTerritorio $ \rbrace $ \\
FK = $ \lbrace $ idSupraTerritorio $ \rbrace $\\
\\
\textbf{ViveEn} $ \lbrace $ \underline{idTerritorio, DNI}, Per\'iodo $ \rbrace $ \\
PK = CK = $ \lbrace $ $ \langle $ idTerritorio, DNI $ \rangle $ $ \rbrace $ \\
FK = $ \lbrace $ idSupraTerritorio $ \rbrace $\\
\\
\textbf{votaEn} $ \lbrace $ \underline{idElecci\'on, idMesa, DNI} $ \rbrace $ \\
PK = CK = $ \lbrace $ $ \langle $ idElecci\'on, idMesa, DNI $ \rangle $ $ \rbrace $ \\
FK = $ \lbrace $ idElecci\'on, idMesa, DNI $ \rbrace $\\
% $ \langle $  $ \rangle $

\end{subsection}

\begin{subsection}{Supuestos Asumidos}

Para la resoluci\'on del trabajo se tomaron los siguientes hechos por sentado:

\begin{itemize}

\item La camioneta solo es conducida por un conductor y este no se modifica en el tiempo.

\item Estamos asumiendo que un presidente de mesa, vicepresidente, tecnico, conductor o fiscal deben ser a su vez votantes.

\end{itemize}

\end{subsection}

\end{section}

	
% ------------ DISCUSION ------------

	%\input{./consultas.tex}
	%\pagebreak

% ------------ CONCLUSIONES ------------

	\begin{section}{Conclusiones}


\end{section}
	\pagebreak

% ------------ MODO USO ------------

	
\begin{section}{Modo de uso}

Para invocar el programa debe irse al directorio \textit{/tleng-tp1/src/formula}. Una vez allí debe ejecutarse el archivo \textit{form.py} escribiendo
\textit{python form.py "formula"} (fórmula deseada entre comillas). Se generará un SVG con el nombre \textit{form.svg} en el mismo directorio.
Por ejemplo: \textit{python form.py} "VIENTO\{FUERTE\}/VELERO\_\{PROBLEMAS\}"

Tenemos dos baterías de tests:

Para correr la primera, dirigirse a \textit{/tleng-tp1/src/tests} y escribir \textit{python testParser.py -v}. Estos tests verifican que los AST (Abstract Syntax Trees) sean generados correctamente a partir de una expresión.

Para correr la segunda, que se encuentra en el mismo directorio que la primera, escribir \textit{python testSVG.py -v}. Estos tests chequean que los los SVG de los casos mostrados en la sección Resultados, sean generados correctamente por el programa. Para ello, corre el programa tomando como input cada una de las expresiones y, posteriormente, compara las imágenes generadas con las imágenes que se encuentran en \textit{/tleng-tp1/src/tests/imgs}, esperando que sean las mismas.

El programa fue testeado en Linux usando la versión 2.7.6 de Python.

\end{section}
	\pagebreak

% ------------ REFERENCIAS ------------

	\begin{thebibliography}{9}
  
%\bibitem{ejemplo}
%\url{http://es.wikipedia.org/wiki/SQLite}

\end{thebibliography}


	\pagebreak

\end{document}
